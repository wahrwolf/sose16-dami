\begin{enumerate}
	\item Für das Datenset Triangle bildet die SOM die Daten in Form einer expontiell wachsenden Sinuskurve ab. In jeder Iteration werden den Datenpunkten je einer BMU und ihrer Nachbarschaft zugeordnet. Diese 'Gewinner' Neuronen, werden dann neu platziert. So bildet sich (je nach Lernfaktor) schnell eine Darstellung der Daten. In  unserem Beispiel ist die grobe Topologie bereits nach 25\% der Interationen erkennbar. Hier findet keine Reduktion der Dimensionen statt.

	\item Für das Datenset Box bildet die SOM die Daten als Gleichförmig verteiltes Netz ab. Analog wie in 1) beschrieben werden auch für die Box-Punkte jeweils ein Neoron mit minimaler Distanz gesucht. Ist dieser gefunden, werden abhängig von der Anzahl der bereits stattgefunden Iterationen die Position der Neuronen angepasst. Je höher die Iteration, desto geringer ist der Einfluss auf die schon bestehende Topology. Dies führt im Beispiel dazu, dass ein Netz erst nach ca 30\% der Iterationen zu erkennen ist, da voher dicht beisammen liegende Punkte mitverschoben werden.
	
	\item Bei dem Clustering durch die SOM, werden bei jeder Berechnung Neuronen in jedem der 4 Eck-Wolken platziert. Da jeder Punkt berücksichtigt wird, ist die Positionierung relativ zuverlässig. Schon nach etwa 25\% der Iteration liegen die Neuronen  zumindest in den Wolken, wodurch die grobe Topology erkennbar ist. Bei dem Clustering durch KMEANS hängt das Ergebnis sehr stark von der Anfangs Positionierung der Centroids ab. So lässt sich bei mehreren Durchläufen sowohl eine gleichmäßige Verteilung der Centroids innerhalb der Punktwolken, als auch eine Positionierung zwischen zwei Wolken sowie mehrfache Centroids innerhalb einer Wolke beobachten. Dafür liegen die Centroids innerhalb einer Wolke zentraler und spiegeln demnach die Verteilung genauer da. Trotzdem liefert die SOM eine stabilere Annäherung.
\end{enumerate}
