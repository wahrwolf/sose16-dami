\begin{enumerate}
	\item Die Figur (..) zeigt das Boxplot-Diagramm von income1 
	\begin{itemize}
		\item Daten für ein Diagram müssen in einer Spalte liegen
		\item Daten müssen nicht sortiert sein
		\item Min- und Max-Werte sind jeweils durch den oberen bzw unteren Querstrich dargestllt
		\item Der Durschnitt (mean) ist der mittlere Abstand der äußeren  Striche
		\item Der Median (also das 50.Perzentil) wird durch den roten Querstricht innerhalb der Box dargestellt
		\item Q1 und Q3 sind jeweils Anfang bzw Ende der blauen Box
		\item um ein Boxplot-Diagram zu erstellen folgt man folgenden Schritten:
		\begin{enumerate}
			\item figure(1)
			\item boxplot(income1)
		\end{enumerate}
	\end{itemize}
	\item foo 

\end{enumerate}
