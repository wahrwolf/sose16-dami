\begin{enumerate}
	\item Die Schwellenwerte ändern sich kaum (Unterschied liegt bei <5\%), wohingengen die Zentren der Cluster sich erkennbar unterscheiden. 

	\item Je höher m gewählt wird, desto höher ist die Akzeptanz anderer Punkte. Bei hohen Werten ist die korrekte Repräsentation nicht mehr gewährleistet, da zu spezifisch. Senkt man m ab, so erhält man eine genauere Repräsentation des Clusters, aber die Grenzen des Clustes werden immer härter

	\item Die Fuzzy-Means Filter offenbaren nicht nicht nur Zugghörigkeiten, sondern auch Tendenzen. So kann z.B. beim Bild "cells" um jede Zelle ein "Ring" beobachtet werden, der im Zweifel eine genauere Zuordnung zu lässt. Im Bild "brain" zeigt der Fuzzy-Filter zusätzlich Strukturen in den einzelnen Bereichen, was eine bessere Analyse (z.B. Flüssigkeit, Dichte) einfacher zu lässt.

\end{enumerate}
